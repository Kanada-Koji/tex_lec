%chapter1_section
%----以下文字の大きさ,フォント,余白などを決定する設定----
\documentclass[a4paper,10pt]{jsarticle}
\bibliographystyle{junsrt} %参考文献の並べ順junsrtの場合,本文内で参照した順
\setlength{\topmargin}{-17truemm}
\setlength{\oddsidemargin}{-0.4truemm}
\setlength{\textwidth}{160truemm}
\setlength{\textheight}{247truemm}
%-------------------------------------------------


\title{セクションの記述}%タイトル
\author{ER18023 金田 康志}%著者
\date{2020年12月27日}%日付

\begin{document}%本文の始まり
%--------------------------------以降に本文を記述--------------------------
\maketitle%上記で設定したtitle,author,dateを表示
\section{セクションについて}
\label{sec:label1}
\subsection{サブセクション}
\label{sec:label2}
\subsubsection{サブサブセクション}
\label{sec:label3}
chapter1\_section.texとchapter1\_section.pdfを照らし合わせて読んでください.\\

texでは「サブサブセクション」までセクションを作成することができます.\\
セクションは,\ref{sec:label1}, \ref{sec:label2}, \ref{sec:label3}のように参照することが可能です.セクションの場合,参照は必要な場合のみ行えばOKです.
また,参照する場合,一度コンパイルしただけだと参照した時に「?」と表示されるので「pLaTeX(ptex2pdf)」→「pLaTeX(ptex2pdf)」と2回コンパイルを実行する必要があります.\\
※ VScodeでの環境構築が完了している場合,「ctrl+s」で保存するだけでコンパイルが完了します.

%-------------------以下課題------------------------
\section{課題}
\noindent 必須:authorを自分の名前,dateを提出日に変更してください.\\
任意:セクションを追加してください(セクションの名前は何でも可).\\

\noindent 編集が完了したら,コンパイルしてください.\\
VScodeで編集している場合,「ctrl+s」で保存&コンパイル→「ctrl+alt+V」で出力されるpdfをプレビューできるので変更点が反映されていることを確認してください.
%----------------------------------------------------

\end{document}